\documentclass[]{article}
\usepackage{lmodern}
\usepackage{amssymb,amsmath}
\usepackage{ifxetex,ifluatex}
\usepackage{fixltx2e} % provides \textsubscript
\ifnum 0\ifxetex 1\fi\ifluatex 1\fi=0 % if pdftex
  \usepackage[T1]{fontenc}
  \usepackage[utf8]{inputenc}
\else % if luatex or xelatex
  \ifxetex
    \usepackage{mathspec}
  \else
    \usepackage{fontspec}
  \fi
  \defaultfontfeatures{Ligatures=TeX,Scale=MatchLowercase}
\fi
% use upquote if available, for straight quotes in verbatim environments
\IfFileExists{upquote.sty}{\usepackage{upquote}}{}
% use microtype if available
\IfFileExists{microtype.sty}{%
\usepackage{microtype}
\UseMicrotypeSet[protrusion]{basicmath} % disable protrusion for tt fonts
}{}
\usepackage[margin=1in]{geometry}
\usepackage{hyperref}
\hypersetup{unicode=true,
            pdftitle={MSc. Research Methods - Statistikteil Lösungen 2018},
            pdfauthor={Gian-Andrea Egeler},
            pdfborder={0 0 0},
            breaklinks=true}
\urlstyle{same}  % don't use monospace font for urls
\usepackage{color}
\usepackage{fancyvrb}
\newcommand{\VerbBar}{|}
\newcommand{\VERB}{\Verb[commandchars=\\\{\}]}
\DefineVerbatimEnvironment{Highlighting}{Verbatim}{commandchars=\\\{\}}
% Add ',fontsize=\small' for more characters per line
\usepackage{framed}
\definecolor{shadecolor}{RGB}{248,248,248}
\newenvironment{Shaded}{\begin{snugshade}}{\end{snugshade}}
\newcommand{\KeywordTok}[1]{\textcolor[rgb]{0.13,0.29,0.53}{\textbf{#1}}}
\newcommand{\DataTypeTok}[1]{\textcolor[rgb]{0.13,0.29,0.53}{#1}}
\newcommand{\DecValTok}[1]{\textcolor[rgb]{0.00,0.00,0.81}{#1}}
\newcommand{\BaseNTok}[1]{\textcolor[rgb]{0.00,0.00,0.81}{#1}}
\newcommand{\FloatTok}[1]{\textcolor[rgb]{0.00,0.00,0.81}{#1}}
\newcommand{\ConstantTok}[1]{\textcolor[rgb]{0.00,0.00,0.00}{#1}}
\newcommand{\CharTok}[1]{\textcolor[rgb]{0.31,0.60,0.02}{#1}}
\newcommand{\SpecialCharTok}[1]{\textcolor[rgb]{0.00,0.00,0.00}{#1}}
\newcommand{\StringTok}[1]{\textcolor[rgb]{0.31,0.60,0.02}{#1}}
\newcommand{\VerbatimStringTok}[1]{\textcolor[rgb]{0.31,0.60,0.02}{#1}}
\newcommand{\SpecialStringTok}[1]{\textcolor[rgb]{0.31,0.60,0.02}{#1}}
\newcommand{\ImportTok}[1]{#1}
\newcommand{\CommentTok}[1]{\textcolor[rgb]{0.56,0.35,0.01}{\textit{#1}}}
\newcommand{\DocumentationTok}[1]{\textcolor[rgb]{0.56,0.35,0.01}{\textbf{\textit{#1}}}}
\newcommand{\AnnotationTok}[1]{\textcolor[rgb]{0.56,0.35,0.01}{\textbf{\textit{#1}}}}
\newcommand{\CommentVarTok}[1]{\textcolor[rgb]{0.56,0.35,0.01}{\textbf{\textit{#1}}}}
\newcommand{\OtherTok}[1]{\textcolor[rgb]{0.56,0.35,0.01}{#1}}
\newcommand{\FunctionTok}[1]{\textcolor[rgb]{0.00,0.00,0.00}{#1}}
\newcommand{\VariableTok}[1]{\textcolor[rgb]{0.00,0.00,0.00}{#1}}
\newcommand{\ControlFlowTok}[1]{\textcolor[rgb]{0.13,0.29,0.53}{\textbf{#1}}}
\newcommand{\OperatorTok}[1]{\textcolor[rgb]{0.81,0.36,0.00}{\textbf{#1}}}
\newcommand{\BuiltInTok}[1]{#1}
\newcommand{\ExtensionTok}[1]{#1}
\newcommand{\PreprocessorTok}[1]{\textcolor[rgb]{0.56,0.35,0.01}{\textit{#1}}}
\newcommand{\AttributeTok}[1]{\textcolor[rgb]{0.77,0.63,0.00}{#1}}
\newcommand{\RegionMarkerTok}[1]{#1}
\newcommand{\InformationTok}[1]{\textcolor[rgb]{0.56,0.35,0.01}{\textbf{\textit{#1}}}}
\newcommand{\WarningTok}[1]{\textcolor[rgb]{0.56,0.35,0.01}{\textbf{\textit{#1}}}}
\newcommand{\AlertTok}[1]{\textcolor[rgb]{0.94,0.16,0.16}{#1}}
\newcommand{\ErrorTok}[1]{\textcolor[rgb]{0.64,0.00,0.00}{\textbf{#1}}}
\newcommand{\NormalTok}[1]{#1}
\usepackage{graphicx,grffile}
\makeatletter
\def\maxwidth{\ifdim\Gin@nat@width>\linewidth\linewidth\else\Gin@nat@width\fi}
\def\maxheight{\ifdim\Gin@nat@height>\textheight\textheight\else\Gin@nat@height\fi}
\makeatother
% Scale images if necessary, so that they will not overflow the page
% margins by default, and it is still possible to overwrite the defaults
% using explicit options in \includegraphics[width, height, ...]{}
\setkeys{Gin}{width=\maxwidth,height=\maxheight,keepaspectratio}
\IfFileExists{parskip.sty}{%
\usepackage{parskip}
}{% else
\setlength{\parindent}{0pt}
\setlength{\parskip}{6pt plus 2pt minus 1pt}
}
\setlength{\emergencystretch}{3em}  % prevent overfull lines
\providecommand{\tightlist}{%
  \setlength{\itemsep}{0pt}\setlength{\parskip}{0pt}}
\setcounter{secnumdepth}{0}
% Redefines (sub)paragraphs to behave more like sections
\ifx\paragraph\undefined\else
\let\oldparagraph\paragraph
\renewcommand{\paragraph}[1]{\oldparagraph{#1}\mbox{}}
\fi
\ifx\subparagraph\undefined\else
\let\oldsubparagraph\subparagraph
\renewcommand{\subparagraph}[1]{\oldsubparagraph{#1}\mbox{}}
\fi

%%% Use protect on footnotes to avoid problems with footnotes in titles
\let\rmarkdownfootnote\footnote%
\def\footnote{\protect\rmarkdownfootnote}

%%% Change title format to be more compact
\usepackage{titling}

% Create subtitle command for use in maketitle
\newcommand{\subtitle}[1]{
  \posttitle{
    \begin{center}\large#1\end{center}
    }
}

\setlength{\droptitle}{-2em}

  \title{MSc. Research Methods - Statistikteil Lösungen 2018}
    \pretitle{\vspace{\droptitle}\centering\huge}
  \posttitle{\par}
    \author{Gian-Andrea Egeler}
    \preauthor{\centering\large\emph}
  \postauthor{\par}
      \predate{\centering\large\emph}
  \postdate{\par}
    \date{November 2018}


\begin{document}
\maketitle

\subsubsection{Übung 4.2S: multiple logistische
Regression}\label{ubung-4.2s-multiple-logistische-regression}

\begin{Shaded}
\begin{Highlighting}[]
\CommentTok{# Genereiert eine Dummyvariable: Fleisch 1, kein Fleisch 0}
\NormalTok{df <-}\StringTok{ }\NormalTok{nova }\CommentTok{# kopiert originaler Datensatz}
\NormalTok{df}\OperatorTok{$}\NormalTok{meat <-}\StringTok{ }\KeywordTok{ifelse}\NormalTok{(nova}\OperatorTok{$}\NormalTok{label_content }\OperatorTok{==}\StringTok{ "Fleisch"}\NormalTok{, }\DecValTok{1}\NormalTok{, }\DecValTok{0}\NormalTok{)}
\NormalTok{df_ <-}\StringTok{ }\NormalTok{df[df}\OperatorTok{$}\NormalTok{label_content }\OperatorTok{!=}\StringTok{ "Buffet"}\NormalTok{, ] }\CommentTok{# entfernt Personen die sich ein Buffet Teller gekauft haben und speichert es in eine neuen Datensatz}

\CommentTok{# Löscht alle Missings bei der Variable "Fleisch"}
\NormalTok{df_ <-}\StringTok{ }\NormalTok{df_[}\OperatorTok{!}\KeywordTok{is.na}\NormalTok{(df_}\OperatorTok{$}\NormalTok{meat), ]}

\CommentTok{#  sieht euch die Verteilung zwischen Fleisch und  kein Fleisch }
\KeywordTok{table}\NormalTok{(df_}\OperatorTok{$}\NormalTok{meat)}
\end{Highlighting}
\end{Shaded}

\begin{verbatim}
## 
##   0   1 
## 387 564
\end{verbatim}

\begin{Shaded}
\begin{Highlighting}[]
\CommentTok{# definiert das logistische Modell und wende es auf den Datensatz an}
\NormalTok{mod0 <-}\StringTok{ }\KeywordTok{glm}\NormalTok{(meat }\OperatorTok{~}\StringTok{ }\NormalTok{gender }\OperatorTok{+}\StringTok{ }\NormalTok{member }\OperatorTok{+}\StringTok{ }\NormalTok{age, }\DataTypeTok{data =}\NormalTok{ df_, }\KeywordTok{binomial}\NormalTok{(}\StringTok{"logit"}\NormalTok{))}
\KeywordTok{summary.lm}\NormalTok{(mod0) }\CommentTok{# Member  und Alter scheinen keinen Einfluss zu nehmen, lassen wir also weg}
\end{Highlighting}
\end{Shaded}

\begin{verbatim}
## 
## Call:
## glm(formula = meat ~ gender + member + age, family = binomial("logit"), 
##     data = df_)
## 
## Weighted Residuals:
##     Min      1Q  Median      3Q     Max 
## -1.6134 -1.0258  0.7174  0.7443  1.1998 
## 
## Coefficients:
##                    Estimate Std. Error t value Pr(>|t|)    
## (Intercept)        0.407608   0.395553   1.030    0.303    
## genderM            0.733556   0.141743   5.175 2.78e-07 ***
## memberStudierende -0.218506   0.197620  -1.106    0.269    
## age               -0.012299   0.009312  -1.321    0.187    
## ---
## Signif. codes:  0 '***' 0.001 '**' 0.01 '*' 0.05 '.' 0.1 ' ' 1
## 
## Residual standard error: 1.002 on 947 degrees of freedom
## Multiple R-squared:  0.001772,   Adjusted R-squared:  -0.00139 
## F-statistic: 0.5603 on 3 and 947 DF,  p-value: 0.6413
\end{verbatim}

\begin{Shaded}
\begin{Highlighting}[]
\CommentTok{# neues Modell ohne Alter und Hochschulzugehörigkeit}
\NormalTok{mod1 <-}\StringTok{ }\KeywordTok{update}\NormalTok{(mod0, }\OperatorTok{~}\NormalTok{. }\OperatorTok{-}\NormalTok{member }\OperatorTok{-}\StringTok{ }\NormalTok{age)}
\KeywordTok{summary.lm}\NormalTok{(mod1)}
\end{Highlighting}
\end{Shaded}

\begin{verbatim}
## 
## Call:
## glm(formula = meat ~ gender, family = binomial("logit"), data = df_)
## 
## Weighted Residuals:
##     Min      1Q  Median      3Q     Max 
## -1.3687 -0.9506  0.7306  0.7306  1.0520 
## 
## Coefficients:
##             Estimate Std. Error t value Pr(>|t|)    
## (Intercept)  -0.1014     0.1128  -0.899    0.369    
## genderM       0.7291     0.1403   5.198 2.47e-07 ***
## ---
## Signif. codes:  0 '***' 0.001 '**' 0.01 '*' 0.05 '.' 0.1 ' ' 1
## 
## Residual standard error: 1.001 on 949 degrees of freedom
## Multiple R-squared:  0.001681,   Adjusted R-squared:  0.0006287 
## F-statistic: 1.598 on 1 and 949 DF,  p-value: 0.2065
\end{verbatim}

\begin{Shaded}
\begin{Highlighting}[]
\CommentTok{# Modeldiagnostik (wenn nicht signifikant, dann OK)}
\DecValTok{1} \OperatorTok{-}\StringTok{ }\KeywordTok{pchisq}\NormalTok{(mod1}\OperatorTok{$}\NormalTok{deviance,mod1}\OperatorTok{$}\NormalTok{df.resid) }\CommentTok{# hochsignifikant, d.h. kein guter Modellfit??}
\end{Highlighting}
\end{Shaded}

\begin{verbatim}
## [1] 5.353906e-11
\end{verbatim}

\begin{Shaded}
\begin{Highlighting}[]
\CommentTok{#Modellgüte (pseudo-R²)}
\DecValTok{1} \OperatorTok{-}\StringTok{ }\NormalTok{(mod1}\OperatorTok{$}\NormalTok{dev }\OperatorTok{/}\StringTok{ }\NormalTok{mod1}\OperatorTok{$}\NormalTok{null) }\CommentTok{# sehr kleines pseudo-R²}
\end{Highlighting}
\end{Shaded}

\begin{verbatim}
## [1] 0.02121993
\end{verbatim}

\begin{Shaded}
\begin{Highlighting}[]
\CommentTok{# Konfusionsmatrix vom  Datensatz}
\CommentTok{# Model Vorhersage}
\NormalTok{predicted <-}\StringTok{ }\KeywordTok{predict}\NormalTok{(mod1, df_, }\DataTypeTok{type =} \StringTok{"response"}\NormalTok{)}

\CommentTok{# erzeugt eine Tabelle mit den beobachteten}
\CommentTok{# Fleischesser/Nichtleischesser und den Vorhersagen des Modells}
\NormalTok{km <-}\StringTok{ }\KeywordTok{table}\NormalTok{(df_}\OperatorTok{$}\NormalTok{meat, predicted }\OperatorTok{>}\StringTok{ }\FloatTok{0.5}\NormalTok{)}
\KeywordTok{dimnames}\NormalTok{(km) <-}\StringTok{ }\KeywordTok{list}\NormalTok{(}
  \KeywordTok{c}\NormalTok{(}\StringTok{"Beobachtung kein Fleisch"}\NormalTok{, }\StringTok{"Beobachtung Fleisch"}\NormalTok{),}
  \KeywordTok{c}\NormalTok{(}\StringTok{"Modell kein Fleisch"}\NormalTok{, }\StringTok{"Modell Fleisch"}\NormalTok{))}
\NormalTok{km}
\end{Highlighting}
\end{Shaded}

\begin{verbatim}
##                          Modell kein Fleisch Modell Fleisch
## Beobachtung kein Fleisch                 166            221
## Beobachtung Fleisch                      150            414
\end{verbatim}

\begin{Shaded}
\begin{Highlighting}[]
\CommentTok{# kalkuliert die Missklassifizierungsrate }
\NormalTok{mf <-}\StringTok{ }\DecValTok{1}\OperatorTok{-}\KeywordTok{sum}\NormalTok{(}\KeywordTok{diag}\NormalTok{(km)}\OperatorTok{/}\KeywordTok{sum}\NormalTok{(km)) }\CommentTok{# ist mit knapp 40% eher hoch}
\NormalTok{mf}
\end{Highlighting}
\end{Shaded}

\begin{verbatim}
## [1] 0.3901157
\end{verbatim}

\paragraph{Methoden}\label{methoden}

Die Kriteriumsvariable ``Fleischkonsum'' ist eine binäre Variable.
Demnach wird eine multiple logistische Regression mit den Prädiktoren
``Alter'', ``Geschlecht'' und ``Hochschulzugehörigkeit'' gerechnet. Die
Modelldiagnostik und das pseudo-\(R^2\) zeigen allerdings, dass das
Modell nicht gut zu den empirischen Daten passt.

\paragraph{Ergebnisse}\label{ergebnisse}

Mit der logistischen Regression kann der Fleischkonsum werder durch das
Geschlecht, die Hochschulzugehörigkeit noch das Alter vorhergesagt
werden. Die Tests für die Modelldiagnostik und das kleine pseudo-\(R^2\)
unterstützen diesen Befund. Auch die hohe Missklassifizierungsrate
(39\%) deutet auf ein Modell, welches nicht zu den Daten passt. Es
sollte nach einem weiteren adäquateren Modell gesucht werden. Bei
näherer Betrachtung der Daten erkennt man, dass einige Personen
wiederholt im Datensatz auftauchen (siehe card\_num). Bei einer weiteren
Analyse müssten die einzelne Individuen ebenfalls berücksichtigt werden
z. B. mit genesteten Modellen (siehe Statistik 5).


\end{document}
