\documentclass[]{book}
\usepackage{lmodern}
\usepackage{amssymb,amsmath}
\usepackage{ifxetex,ifluatex}
\usepackage{fixltx2e} % provides \textsubscript
\ifnum 0\ifxetex 1\fi\ifluatex 1\fi=0 % if pdftex
  \usepackage[T1]{fontenc}
  \usepackage[utf8]{inputenc}
\else % if luatex or xelatex
  \ifxetex
    \usepackage{mathspec}
  \else
    \usepackage{fontspec}
  \fi
  \defaultfontfeatures{Ligatures=TeX,Scale=MatchLowercase}
\fi
% use upquote if available, for straight quotes in verbatim environments
\IfFileExists{upquote.sty}{\usepackage{upquote}}{}
% use microtype if available
\IfFileExists{microtype.sty}{%
\usepackage{microtype}
\UseMicrotypeSet[protrusion]{basicmath} % disable protrusion for tt fonts
}{}
\usepackage{hyperref}
\hypersetup{unicode=true,
            pdftitle={Research Methods},
            pdfborder={0 0 0},
            breaklinks=true}
\urlstyle{same}  % don't use monospace font for urls
\usepackage{longtable,booktabs}
\usepackage{graphicx,grffile}
\makeatletter
\def\maxwidth{\ifdim\Gin@nat@width>\linewidth\linewidth\else\Gin@nat@width\fi}
\def\maxheight{\ifdim\Gin@nat@height>\textheight\textheight\else\Gin@nat@height\fi}
\makeatother
% Scale images if necessary, so that they will not overflow the page
% margins by default, and it is still possible to overwrite the defaults
% using explicit options in \includegraphics[width, height, ...]{}
\setkeys{Gin}{width=\maxwidth,height=\maxheight,keepaspectratio}
\IfFileExists{parskip.sty}{%
\usepackage{parskip}
}{% else
\setlength{\parindent}{0pt}
\setlength{\parskip}{6pt plus 2pt minus 1pt}
}
\setlength{\emergencystretch}{3em}  % prevent overfull lines
\providecommand{\tightlist}{%
  \setlength{\itemsep}{0pt}\setlength{\parskip}{0pt}}
\setcounter{secnumdepth}{5}
% Redefines (sub)paragraphs to behave more like sections
\ifx\paragraph\undefined\else
\let\oldparagraph\paragraph
\renewcommand{\paragraph}[1]{\oldparagraph{#1}\mbox{}}
\fi
\ifx\subparagraph\undefined\else
\let\oldsubparagraph\subparagraph
\renewcommand{\subparagraph}[1]{\oldsubparagraph{#1}\mbox{}}
\fi

%%% Use protect on footnotes to avoid problems with footnotes in titles
\let\rmarkdownfootnote\footnote%
\def\footnote{\protect\rmarkdownfootnote}

%%% Change title format to be more compact
\usepackage{titling}

% Create subtitle command for use in maketitle
\providecommand{\subtitle}[1]{
  \posttitle{
    \begin{center}\large#1\end{center}
    }
}

\setlength{\droptitle}{-2em}

  \title{Research Methods}
    \pretitle{\vspace{\droptitle}\centering\huge}
  \posttitle{\par}
    \author{}
    \preauthor{}\postauthor{}
      \predate{\centering\large\emph}
  \postdate{\par}
    \date{2019-08-15}

\usepackage{booktabs}
\usepackage{amsthm}
\makeatletter
\def\thm@space@setup{%
  \thm@preskip=8pt plus 2pt minus 4pt
  \thm@postskip=\thm@preskip
}
\makeatother

\AtBeginDocument{\let\maketitle\relax} % supress header


\usepackage{fancyhdr}

\pagestyle{fancy}
% \fancyhead[CO,CE]{ZHAW LSFM}
\fancyfoot[CO,CE]{\textsc{zhaw lsfm}}
\fancyfoot[LE,RO]{\thepage}
\fancyfoot[RE,LO]{Modul \textit{ResearchMethods}}

\begin{document}
\maketitle

\newgeometry{tmargin=1.5cm,lmargin=2.5cm,rmargin=2.5cm,bmargin=0.5cm} %verbose

\begin{titlepage}
\begin{center}
  
{\small 
ZURICH UNIVERSITY OF APPLIED SCIENCES
\linebreak SCHOOL OF LIFE SCIENCES AND FACILITY MANAGEMENT
\linebreak INSTITUTE OF NATURAL RESOURCE SCIENCES
}

\end{center}
\vspace{1.5cm}
\begin{center}

{\Large Übungsunterlagen und Demoscript zum Modul \emph{Research Methods}}

\end{center}
 \vspace{1cm}

% \begin{figure}[htbp]
%   \centering
%   \includegraphics[width=1\textwidth]{Images/Reh-flucht-compilation3_small.png}
%   \label{titelbild} 
% \end{figure}

\begin{center}
\textbf{Herbstsemester 2018}

%\textbf{Patrick Laube und Nils Ratnaweera}

\end{center} 

\vspace{1.0cm}


\newpage
\thispagestyle{empty}
\begin{minipage}{15cm}
\begin{flushleft}




\vspace{18cm}
{\large Impressum:}

\vspace{0.5cm}

% \textbf{Citation:} 
Laube, P. et al. (2018): Übungsunterlagen für das Modul Research Methods im Master of Science in Umwelt und Natürliche Ressourcen. IUNR, Zürcher Hochschule der Angewandten Wissenschaften (ZHAW), W{\"a}denswil.

\end{flushleft}
\end{minipage}

\end{titlepage}
\restoregeometry

{
\setcounter{tocdepth}{1}
\tableofcontents
}
\chapter{Einleitung}\label{einleitung}

Das Modul „Research Methods`` vermittelt vertiefte Methodenkompetenzen
für praxisorientiertes und angewandtes wissenschaftliches Arbeiten im
Fachbereich „Umwelt und Natürliche Ressourcen`` auf MSc-Niveau. Die
Studierenden erarbeiten sich vertiefte Methodenkompetenzen für die
analytische Betrachtung der Zusammenhänge im Gesamtsystem „Umwelt und
Natürliche Ressourcen``. Die Studierenden erlernen die methodischen
Kompetenzen, auf denen die nachfolgenden Module im MSc Programm UNR
aufbauen. Das Modul vermittelt einerseits allgemeine,
fächerübergreifende methodische Kompetenzen (z.B. Wissenschaftstheorie,
computer-gestützte Datenverar-beitung und Statistik).

Auf dieser Plattform (RStudio Connect) werden die Unterlagen für die
R-Übungsteile bereitgestellt. Es werden sukzessive sowohl Demo-Files,
Aufgabenstellungen und Lösungen veröffentlicht.


\end{document}
